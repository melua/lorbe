\section{\sc{\textit{Les révélations}}\hfill}

%%%%%%%%%%%%%%%%%%%%%%%%%%%%%
% LXIV. LA CONNAISSANCE
%%%%%%%%%%%%%%%%%%%%%%%%%%%%%
\stepcounter{compteur}
\poemtitle{\numero La connaissance}
\settowidth{\versewidth}{D'où jaillissent les caresses et le bonheur}
\begin{verse}[\versewidth]
Les jambes croisées délicatement fines \\
Les yeux brillant à l'appel du savoir : \\
Tu écris assidûment ces lignes \\
Comme je m'abreuve de te voir

Parfois tu te penches sur le papier \\
Laissant tes cheveux noirs de jais \\
Dévoiler une nuque blanche à souhait \\
Où viennent se perdre mes pensées

Alors tes mains sont un bouquet de fleur \\
D'où jaillissent les caresses et le bonheur \\
Les mots de ta tendresse esseulée \\
Les joues de qui vient t'embrasser

Et de tes beaux yeux les longs cils \\
Sont les battements de ton profil \\
Des écueils où viennent s'échouer \\
Souvent par milliers mes baisers

Et ta bouche et tes lèvres \\
Qu'ouvre un large sourire \\
Que ferme un pieux soupir \\
Sont ma santé et ma fièvre

\newpage

Dans ces instants d'absence \\
Où me fait rougir ton sérieux \\
La seule vérité à ma connaissance \\
C'est que de toi je suis amoureux !
\end{verse}

%\newpage

%%%%%%%%%%%%%%%%%%%%%%%%%%%%%
% LXV. LE VISAGE
%%%%%%%%%%%%%%%%%%%%%%%%%%%%%
\stepcounter{compteur}
\poemtitle{\numero Le visage}
\settowidth{\versewidth}{Et cette fièvre absente peut m'étouffer :}
\begin{verse}[\versewidth]
%Les jours passent comme des heures, \\
%A la mesure du malheur \\
%De ne t'avoir ici présent, \\
%L'amour nous fait gagner du temps !

Bien habile et clairvoyant \\
Qui trouverait ma princesse, \\
La raison de ma détresse \\
Et de mon feu incessant :

%C'est un vin enivrant que de l'aimer, \\
%En vain j'ai essayé de m'y soustraire \\
%Mais sans cette faible alliance j'erre \\
%Et cette fièvre absente peut m'étouffer :

Elle est mon souffle et mes cendres \\
Le vent d'espoir qui porte mes ailes \\
Elle est ma source, elle est mes peines \\
Et mon cœur à ses yeux vient se pendre !

Elle a le visage de la poésie, \\
A la fois forte et fragile, \\
Indispensable et inutile, \\
Une pauvreté qui enrichit !

%Les jours comme les heures, \\
%A la mesure du malheur \\
%Et du temps passé \\
%A te conter...
\end{verse}

\newpage

%%%%%%%%%%%%%%%%%%%%%%%%%%%%%
% LXVI. TA CHEVELURE
%%%%%%%%%%%%%%%%%%%%%%%%%%%%%
\stepcounter{compteur}
\poemtitle{\numero Ta Chevelure}
\settowidth{\versewidth}{Au-dessus de toi la lune brille aux éclats}
\begin{verse}[\versewidth]
Dans le parfum libérés par tes mains \\
Tes cheveux comme une pluie d'étoile \\
Viennent effleurer ta peau de blanc satin \\
Qui frémit et rit de cette douce caresse

Au-dessus de toi la lune brille aux éclats \\
Joue de ta chevelure comme d'un voile \\
Et c'est la nuit dans ton sourire de soie \\
Qui frémit et rit de cette douce caresse.
\end{verse}

%\newpage

%%%%%%%%%%%%%%%%%%%%%%%%%%%%%
% LXVII. LA DECLARATION
%%%%%%%%%%%%%%%%%%%%%%%%%%%%%
\stepcounter{compteur}
\poemtitle{\numero La Déclaration}
\settowidth{\versewidth}{Je t'aime ! Je t'aime ! Je t'aime ! Je t'aime !}
\begin{verse}[\versewidth]
Parce que c'est un monstre de charme \\
Qui a dévoré mes journées, \\
Sous son regard je suis sans armes : \\
Je peux seulement aimer…

Puisque c'est une pensée sans trêve \\
Il faut bien avouer mes rêves, \\
A la nymphe ce doux poème : \\
Je t'aime ! Je t'aime ! Je t'aime ! Je t'aime !
\end{verse}

\newpage

%%%%%%%%%%%%%%%%%%%%%%%%%%%%%
% LXVIII. NUIT BLANCHE
%%%%%%%%%%%%%%%%%%%%%%%%%%%%%
\stepcounter{compteur}
\poemtitle{\numero Nuit Blanche}
\settowidth{\versewidth}{Ici la nuit ne se couchera pas de sitôt,}
\begin{verse}[\versewidth]
Ici la nuit ne se couchera pas de sitôt, \\
Elle est claire d'une lueur blafarde : \\
Si la clarissime brille par mégarde, \\
Mon cœur lui ne connaît pas de repos.
\end{verse}

%\newpage

%%%%%%%%%%%%%%%%%%%%%%%%%%%%%
% LXIX. NUIT DE CRISTAL
%%%%%%%%%%%%%%%%%%%%%%%%%%%%%
\stepcounter{compteur}
\poemtitle{\numero Nuit de Cristal}
\settowidth{\versewidth}{Vide, claire, pleine de lumière,}
\begin{verse}[\versewidth]
%La nuit était chaude \\
%La lune au zénith \\
%J'écrivais une ode : \\
%L'amour sans limite.

%Dernier verre avant le sommeil \\
%Mais un soudain rire de cristal \\
%Fendit le silence de ma veille \\
%Figeant de peur toute la salle…

Déjà des ombres sylvestres \\
Brûlaient l'obscurité de cimes \\
Recouvrant de suie mon être : \\
Je sombrais parmi les rimes.

Dernière aire pour ma vie, \\
Vide, claire, pleine de lumière, \\
Une clairière sans soucis \\
Où j'ai fini mon vers !
\end{verse}

\newpage

%%%%%%%%%%%%%%%%%%%%%%%%%%%%%
% LXX. LES CHATIMENTS D'UN CœUR
%%%%%%%%%%%%%%%%%%%%%%%%%%%%%
\stepcounter{compteur}
\poemtitle{\numero Les châtiments d'un cœur}
\settowidth{\versewidth}{Mais le rêve était plus enivrant que l'altitude}
\begin{verse}[\versewidth]
A la recherche d'une source, auguste Claire \\
J'ai tenté en vain de confondre mes chimères. \\
Mais le rêve était plus enivrant que l'altitude \\
Et je devais bientôt regretter mon attitude :

La vérité m'a ouvert le corps ! \\
Et mon cœur dans les abysses \\
Râle et expire des remords \\
Quitte et lentement glisse :

Ne pas te revoir sera pire que la mort \\
Et tout ce que j'ai pu éprouver. \\
Cela sera -- hélas ! -- le triste sort \\
Que je devrais endurer :

Sans toi et sans pleurs \\
Je souffrirais quand même \\
Les châtiments d'un cœur \\
Trop plein de ce qu'il aime.
\end{verse}

\newpage

%%%%%%%%%%%%%%%%%%%%%%%%%%%%%
% LXXI. CONFESSION D'UN CLERC
%%%%%%%%%%%%%%%%%%%%%%%%%%%%%
\stepcounter{compteur}
\poemtitle{\numero Confession d'un clerc}
\settowidth{\versewidth}{On s'initie aux mystères de sa vertu}
\begin{verse}[\versewidth]
Il y a je ne sais quoi qui fascine \\
Dans sa taille et ses jambes fines, \\
On s'initie aux mystères de sa vertu \\
Dans ses moindres échos charnus.

On est tout entier attaché \\
A ses charmes révélés, \\
Mais on reconnaît là en elle \\
Une triste religion personnelle.
\end{verse}

%\newpage

%%%%%%%%%%%%%%%%%%%%%%%%%%%%%
% LXXII. LA MUSE
%%%%%%%%%%%%%%%%%%%%%%%%%%%%%
\stepcounter{compteur}
\poemtitle{\numero La Muse}
\settowidth{\versewidth}{Te répand dans villes et contrées}
\begin{verse}[\versewidth]
Comme un trait dans la nuit \\
Une étoile dans le noir infini \\
Un ballet céleste à son apogée \\
Une éclipse de journée

Tu recouvre d'émois les années \\
Te répand dans villes et contrées \\
Disperse ta beauté dans mes vies \\
Fait chanter les lys.
\end{verse}

\newpage

%%%%%%%%%%%%%%%%%%%%%%%%%%%%%
% LXXIII. LE JOUR
%%%%%%%%%%%%%%%%%%%%%%%%%%%%%
\stepcounter{compteur}
\poemtitle{\numero Le Jour}
\settowidth{\versewidth}{Tomberont ci et là sur le sol oublié du soleil}
\begin{verse}[\versewidth]
Le vent soufflera doucement dans l'allée \\
Des feuilles pareilles à mon âme desséchée \\
Tomberont ci et là sur le sol oublié du soleil \\
Je me recouvrirai -- pourtant sans sommeil

De neige et d'agonie \\
Je serais gisant et sans vie \\
Sans inspiration, sans air parfumé \\
Le jour où j'ai cessé d'aimer.
\end{verse}

%\newpage

%%%%%%%%%%%%%%%%%%%%%%%%%%%%%
% LXXIV. UNE INCONNUE
%%%%%%%%%%%%%%%%%%%%%%%%%%%%%
\stepcounter{compteur}
\poemtitle{\numero Une inconnue}
\settowidth{\versewidth}{Dans une gerbe de folie et de sentiments,}
\begin{verse}[\versewidth]
Il descend doucement la rue, \\
Fier, insoumis parmi les cœurs conquis, \\
Quand son regard croise une inconnue \\
Il tombe à la renverse et pâlit.

Dans une gerbe de folie et de sentiments, \\
Battu par l'appel des yeux, \\
Il courbe l'échine si rapidement \\
Le cœur qui est amoureux !
\end{verse}

\newpage

%%%%%%%%%%%%%%%%%%%%%%%%%%%%%
% LXXV. LE PROFANE
%%%%%%%%%%%%%%%%%%%%%%%%%%%%%
\stepcounter{compteur}
\poemtitle{\numero Le Profane}
\settowidth{\versewidth}{Nulles choses pareilles pour nous faire cesser d'aimer !}
\begin{verse}[\versewidth]
A tant de beautés qu'en ce monde on peut trouver \\
Je comprends très bien qu'on ne puisse plaire \\
Mais fasse le ciel fassent les enfers \\
Qu'une au moins puisse m'aimer !

Vie et mort pourraient se fondre et le temps s'arrêter \\
Nos deux âmes partiraient se confondre dans l'éternité \\
Où sont nulle aube nul soleil sur nos corps élevés \\
Nulles choses pareilles pour nous faire cesser d'aimer !
\end{verse}

\newpage

%%%%%%%%%%%%%%%%%%%%%%%%%%%%%
% LXXVI. TOI QUI RIS
%%%%%%%%%%%%%%%%%%%%%%%%%%%%%
\stepcounter{compteur}
\poemtitle{\numero Toi qui ris}
\settowidth{\versewidth}{Et mon corps vibre, tu le rends fou,}
\begin{verse}[\versewidth]
Toi qui délies tes cheveux \\
Noie tous nos repères \\
Dans la galaxie de tes yeux \\
Où milles étoiles se perdent

Toi qui a allumé les feux \\
Qui de loin nous appellent, \\
Nous consument, nous rendent preux, \\
Nous occultent le soleil

Toi qui a semé l'ivresse \\
Dans nos cœurs concentrés \\
Souffle un vent de détresse \\
Dans nos corps tourmentés

Toi qui ris, aies pitié de nous, \\
J'ai le cœur ivre qui s'étiole \\
Et mon corps vibre, tu le rends fou, \\
Mon amour meurt et s'envole…
\end{verse}

\newpage

%%%%%%%%%%%%%%%%%%%%%%%%%%%%%
% LXXVII. A TRAVERS
%%%%%%%%%%%%%%%%%%%%%%%%%%%%%
\stepcounter{compteur}
\poemtitle{\numero A travers}
\settowidth{\versewidth}{A travers les vallées et les plaines,}
\begin{verse}[\versewidth]
A travers l'au-delà \\
Le revers du papier \\
Imprimera ma foi \\
Pour l'éternité :

A travers les vagues de l'océan \\
Où tu me donneras la vie \\
Dans tes yeux flamboyants \\
Je découvrirai ma patrie.

A travers les vallées et les plaines, \\
Dans des masses de terreur, \\
Où mon cœur sera en peine, \\
Je déposerai le sang et les pleurs.

A travers les nuées d'étoiles \\
A travers les chansons \\
Les deuils et les voiles \\
Je crierai ton nom !
\end{verse}

\newpage

%%%%%%%%%%%%%%%%%%%%%%%%%%%%%
% LXXVIII. ELEGIE
%%%%%%%%%%%%%%%%%%%%%%%%%%%%%
\stepcounter{compteur}
\poemtitle{\numero Élégie}
\settowidth{\versewidth}{Une brindille qui tourne et chancelle}
\begin{verse}[\versewidth]

\emph{\hspace{9em}Suffira de verser quelques pleurs \\
\hspace{9em}Pour arroser vos propres fleurs \\
\hspace{9em}De méninges \\}
\attrib{G. Moustaki}

A l'éveil du recueil, une étincelle, \\
Un fusain sur une page vierge, \\
Une brindille qui tourne et chancelle \\
Je la chérie, je la submerge,

Je chavire à l'émoi ! \\
Tremblante comme une feuille, \\
Elle est posée sous mes doigts, \\
Pâle comme un linceul.
\end{verse}

\newpage

%%%%%%%%%%%%%%%%%%%%%%%%%%%%%
% LXXIX. LA PASTORALE
%%%%%%%%%%%%%%%%%%%%%%%%%%%%%
\stepcounter{compteur}
\poemtitle{\numero La Pastorale}
\settowidth{\versewidth}{Une brindille qui tourne et chancelle}
\begin{verse}[\versewidth]
Dans ce champ entre deux collines \\
Où l'on aime à s'allonger \\
Parmi les ombres qui s'inclinent \\
Des bottes de pailles ficelées

Le vent du ciel dans tes cheveux \\
Couche les herbes avec douceur \\
Et les étoiles dans tes yeux \\
Bercent mes bras avec ferveur

Dans une mouvance enchevêtrée \\
Deux érudits les langues déliées \\
Expriment avec calme et volupté \\
Le langage de l'Humanité.
\end{verse}

%\newpage

%%%%%%%%%%%%%%%%%%%%%%%%%%%%%
% LXXXX. AU GRE DU VENT
%%%%%%%%%%%%%%%%%%%%%%%%%%%%%
\stepcounter{compteur}
\poemtitle{\numero Au gré du vent}
\settowidth{\versewidth}{Souffle sur les joues par ses pensers,}
\begin{verse}[\versewidth]
Tes os ne sont pas de cristal, \\
Il te faut lâcher prise ! \\
Ton cœur -- farouche animal \\
Pareil à la brise

Souffle sur les joues par ses pensers, \\
Ses proses, ses poèmes, \\
Sans jamais pourvoir les toucher, \\
Sans pouvoir dire je t'aime.
\end{verse}