\section{\sc{\textit{Ruines d'un printemps nommé amour}}\hfill}

%%%%%%%%%%%%%%%%%%%%%%%%%%%%%
% I. L'AMOUR DE NOS JOURS
%%%%%%%%%%%%%%%%%%%%%%%%%%%%%
\stepcounter{compteur}
\poemtitle{\numero L'amour de nos jours}
\settowidth{\versewidth}{Ô douleur ! ô douleur ! Le temps dévore sa flamme.}
\begin{verse}[\versewidth]

\emph{\hspace{15em}Un éclair... puis la nuit ! \\}
\attrib{C. Baudelaire}

Quand un homme aime une femme \\
Son cœur est brûlé par la passion \\
Mais le cours de la vie finit par avoir raison. \\
Ô douleur ! ô douleur ! Le temps dévore sa flamme.

Quand au coucher du soleil \\
Les lumières vermeilles des cieux jaillissent \\
Mille peines les cœurs subissent. \\
Telle est l'épreuve ultime du triste réveil.

Nature ! Pourquoi quand le ciel se fait noir \\
Nous, jeunes hommes, braves aimants \\
En rêves et souhaits cependant \\
Devons-nous souffrir le schisme du soir ?

%Nature ! Se pourrait-il que de tant de force \\
%Tu nous ais fait tributaires \\
%Et de toutes les puissances de la Terre \\
%Nous ayons hérité de celle qui brise l'écorce ?

Ô femme ! Temple de l'amour ! \\
Griffes, dents et muscles ne font pas le poids \\
Face aux poèmes, aux mots, à cette voix \\
Que pour toi porte à nos lèvres l'amour !

%Ô mienne ! J'aurais aimé la Terre mon toit ! \\
%Renié tous les paradis \\
%Si tu m'avais dit \\
%Le paradis est bien sur Terre, avec toi !
\end{verse}

\newpage

%%%%%%%%%%%%%%%%%%%%%%%%%%%%%
% II. RUINES D'UN PRINTEMPS
%%%%%%%%%%%%%%%%%%%%%%%%%%%%%
\stepcounter{compteur}
\poemtitle{\numero Ruines d'un printemps}
\settowidth{\versewidth}{Comme toi et ta beauté qui m'a fuit.}
\begin{verse}[\versewidth]
\indentpattern{0024000}
\begin{patverse}
Les feuilles mortes tombaient \\
Une main m'a frôlé \\
Un sourire m'a illuminé \\
Une feuille a glissé.

Et des cendres d'un amour passé \\
Une volute de fumée s'est enfuit \\
Comme toi et ta beauté qui m'a fuit.
\end{patverse}
\end{verse}

%\newpage

%%%%%%%%%%%%%%%%%%%%%%%%%%%%%
% III. ET POURTANT
%%%%%%%%%%%%%%%%%%%%%%%%%%%%%
\stepcounter{compteur}
\poemtitle{\numero Et pourtant}
\settowidth{\versewidth}{Les lignes blanches dansaient comme dans un rêve}
\begin{verse}[\versewidth]
Et pourtant ta beauté suit mon âme \\
Comme dans un rêve debout \\
La piste se déroule, les arbres filent, \\
Et devant mes yeux perdus \\
Je ne vois que toi.

L'ébauche d'un sentiment passé refait surface \\
Celle que j'avais oubliée et qui me revient \\
Les lignes blanches dansent comme dans un rêve \\
Et je sombre dans le néant bordé de tes lèvres, \\
Fruits du bonheur m'interpellant, \\
En salutation amère s'ouvrant.

Les lignes blanches dansaient comme dans un rêve \\
Et peu à peu j'en sortais \\
Parcouru \\
D'un frisson \\
A l'idée que tu sois passé... \\
Les arbres filaient devant mes yeux.
\end{verse}

%\newpage

%%%%%%%%%%%%%%%%%%%%%%%%%%%%%
% IV. LE PRINTEMPS DES PASSIONS
%%%%%%%%%%%%%%%%%%%%%%%%%%%%%
%\stepcounter{compteur}
%\poemtitle{\numero Le printemps des passions}
%\settowidth{\versewidth}{M'exclamant "Cet empire enfin si grand, si glorieux,}
%\begin{verse}[\versewidth]
%Si quelque dieu que ce soit \\
%S'adressant à moi \\
%Me faisait héritier de l'univers \\
%De la mer jusqu'au Rhin \\
%J'y renoncerais avec joie \\
%Car rien ne vaut le poison qui découle \\
%De tes yeux verts du paradis \\
%Et prendrais à témoin le sage portrait \\
%Du mage Hugo sur son cadre \\
%M'exclamant ``Cet empire enfin si grand, si glorieux, \\
%N'est pas de vos présents le plus cher à mes yeux'' \\
%Au grand-poète de rajouter ``Aimer est de l'Homme, \\
%Gouverner les Cieux est de Dieu'' \\
%Et sa voix se mêlerait dans mon chœur \\
%Au chant du Min Imperator Mundi. \\
%\end{verse}

%\newpage

%%%%%%%%%%%%%%%%%%%%%%%%%%%%%
% V. PAR DEFAUT
%%%%%%%%%%%%%%%%%%%%%%%%%%%%%
%\stepcounter{compteur}
%\poemtitle{\numero Par défaut}
%\settowidth{\versewidth}{Une plume et un papier pour se faire pardonner.}
%\begin{verse}[\versewidth]
%Avarice pour posséder \\
%Gourmandise pour le plaisir \\
%Défaitisme pour abandonner \\
%Paresse pour dormir \\
%Désir pour aimer \\
%Egoïsme pour mentir \\
%Supériorité pour gouverner \\
%Colère pour affermir \\
%Intolérance pour corriger \\
%Jalousie pour tout détruire \\
%Haine pour tuer \\
%Solitude pour écrire \\
%Une plume et un papier pour se faire pardonner.
%\end{verse}

%\newpage

%%%%%%%%%%%%%%%%%%%%%%%%%%%%%
% VI. AUBE RUISSELANTE
%%%%%%%%%%%%%%%%%%%%%%%%%%%%%
\stepcounter{compteur}
\poemtitle{\numero Aube ruisselante}
%\settowidth{\versewidth}{Parcouru ses formes jusqu'au creux de ses reins}
\begin{verse}%[\versewidth]
\center{
J'ai embrassé l'aube d'hiver. \\
Déposé un baiser sur ses lèvres glacées \\
Parcouru son corps de mes mains \\
Caressé la toison d'or de mes doigts \\
Senti sa poitrine se dresser sous mon joug \\
Et sa voix vibrer dans mon cou \\
En entendant la neige tomber sur les toits \\
En entendant son cœur battre dans son sein \\
Et son corps de délice frémir \\
Senti son parfum d'essence de Guerlain \\
Caressé sa chevelure dorée de mon souffle \\
Parcouru ses formes jusqu'au creux de ses reins \\
Déposé un baiser sur ses lèvres trempées \\
J'ai embrassé l'aube d'hiver.

%Je l'ai enlacé longtemps \\
%Cet être du ciel \\
%Qui est bien moins cruel \\
%Que l'aube élancée du printemps.
}
\end{verse}

%\newpage

%%%%%%%%%%%%%%%%%%%%%%%%%%%%%
% VII. LE PUZZLE
%%%%%%%%%%%%%%%%%%%%%%%%%%%%%
%\stepcounter{compteur}
%\poemtitle{\numero Le puzzle}
%\settowidth{\versewidth}{Elle est souvent sous nos yeux mais on ne la voit pas toujours.}
%\begin{verse}[\versewidth]
%Souvent, pour se distraire, les hommes \\
%Prennent des puzzles, images divisées \\
%Qu'il faut réunifier sous les mêmes couleurs, \\
%Pareil à milles royaumes pour un empire.

%A peine les ont-ils terminés \\
%Que ces créateurs, repus et satisfaits, \\
%Laissent dans l'oubli ces tableaux morcelés \\
%Comme on abandonne les chiens en été.

%Cette œuvre, comme elle est belle et poussiéreuse ! \\
%Elle, redécouverte par les mains humaines qui \\
%L'ont reconstitué avec la force du temps, elle, \\
%L'autre façon de répondre aux couleurs qui nous entourent !

%L'Homme est semblable à un puzzle immense \\
%Dont la dernière pièce s'appelle l'amour. \\
%Mais c'est un jeu de patience, \\
%Elle est souvent sous nos yeux mais on ne la voit pas toujours.
%\end{verse}

%\newpage

%%%%%%%%%%%%%%%%%%%%%%%%%%%%%
% VIII. LUCIDE
%%%%%%%%%%%%%%%%%%%%%%%%%%%%%
\stepcounter{compteur}
\poemtitle{\numero Lucide}
\settowidth{\versewidth}{Toute résistance à cet amour est vaine.}
\begin{verse}[\versewidth]
Cheveux couleur amour \\
Robes de satin, de velours \\
Visage né dans la lumière \\
Yeux diamants de rivière \\
Beauté sublimée \\
Gorge à croquer \\
Profusion de passion dans mes veines \\
Toute résistance à cet amour est vaine \\
Il suffirait de presque rien \\
Peut-être la peur en moins \\
Pour que je lui dise \\
Je t'aime.
\end{verse}