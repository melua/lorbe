%%%%%%%%%%%%%%%%%%%%%%%%%%%%%
% PREFACE
%%%%%%%%%%%%%%%%%%%%%%%%%%%%%
\chapter*{Préface}
Ce recueil de poèmes est composé de deux parties, conçues à l'origine comme des recueils indépendants et eux-même divisés en trois chapitres ; la première intitulée \emph{Le Schisme du soir} a été écrite entre 2003 et 2005 pendant mes années lycées ; la seconde nommée \emph{L'Amour en doute} a été écrite entre 2005 et 2008 lorsque j'étais en faculté. Bien que les premiers poèmes soit de mon propre aveu brouillons et enfantins, il m'est apparu nécessaire de les laisser pour en comprendre le dénouement : \emph{L'Orbe} est un chemin de vie, une évolution de pensées. %Quant au court récit qui lui fait suite, \emph{La Légende de Tethys}, il s'agit d'un conte fantastique écrit entre 2008 et 2011 puis totalement remanié en 2013.

\vfill
\begin{flushright}
\emph{Je dédie ce recueil à L. Helleringer,\\
très cher compagnon de poésie.}
\end{flushright}

%%%%%%%%%%%%%%%%%%%%%%%%%%%%%
% INTRODUCTION
%%%%%%%%%%%%%%%%%%%%%%%%%%%%%

\chapter*{Introduction}
Qu'une chose soit difficile doit nous être une raison de plus de nous y tenir. Il est bon aussi d'aimer ; car l'amour est difficile. L'amour d'un humain pour un autre, c'est peut-être l'épreuve la plus difficile pour chacun de nous, c'est le plus haut témoignage de nous même ; l'œuvre suprême dont toutes les autres ne sont que les préparations. C'est pour cela que les êtres jeunes, neufs en toutes choses, ne savent pas encore aimer ; ils doivent apprendre. De toutes les forces de leur être, concentrées dans leur cœur qui bat anxieux et solitaire, ils apprennent à aimer. Tout apprentissage est un temps de clôture. Ainsi pour celui qui aime, l'amour n'est longtemps, et jusqu'au large de la vie, que solitude, solitude toujours plus intense et plus profonde. L'amour ce n'est pas dès l'abord se donner, s'unir à un autre. (Que serait l'union de deux êtres encore imprécis, inachevés, dépendants ?) L'amour, c'est l'occasion unique de mûrir, de prendre forme, de devenir soi-même un monde pour l'amour de l'être aimé. C'est une haute exigence, une ambition sans limite, qui fait de celui qui aime un élu qu'appelle le large.
\begin{flushright}
(\sc{Rainer-Maria Rilke : Lettres à un jeune poète, VII})
\end{flushright}