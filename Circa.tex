\section{\sc{\textit{Circa mea pectora}}\hfill}

%%%%%%%%%%%%%%%%%%%%%%%%%%%%%
% LV. LA PRUNELLE DE MES CIEUX
%%%%%%%%%%%%%%%%%%%%%%%%%%%%%
\stepcounter{compteur}
\poemtitle{\numero La prunelle de mes cieux}
\settowidth{\versewidth}{Glissant parmi les pierres et dans les cœurs.}
\begin{verse}[\versewidth]
Elle brille en moi comme milles aurores, \\
Un voile blanc aux reflets cousus d'or, \\
Limpide comme l'eau de roche, \\
Écumant les rivages proches.

%Elle vibre en moi comme une rivière, \\
%Un lit de cheveux aux grands yeux sincères, \\
%Désespérément belle dans toute sa candeur, \\
%Glissant parmi les pierres et dans les cœurs.

Elle chante en moi comme un cri dément, \\
Une paix indolente au visage charmant, \\
Amer désert d'absence sans bruit \\
Des rêves rebondissant dans mon esprit.

J'ai longtemps vécu dans ces collines \\
Offrant ma vie en trésor \\
A ma perle blanche et câline \\
J'ai vécu et j'y vis encore.
\end{verse}

\newpage

%%%%%%%%%%%%%%%%%%%%%%%%%%%%%
% LVI. REQUIEM
%%%%%%%%%%%%%%%%%%%%%%%%%%%%%
\stepcounter{compteur}
\poemtitle{\numero Requiem}
\settowidth{\versewidth}{De ma plume tes milles baisers,}
\begin{verse}[\versewidth]
L'ombre lancinante \\
Du papier sur le lit \\
Projette force violente \\
Pour dire que tu es partie.

Le fauteuil boiteux, \\
Où tu flânais quelques fois, \\
Se souvient des jours heureux \\
Du trône et son roi.

Le sot que j'étais \\
Pleure le spectre \\
De ce que j'aimais \\
Et ne semble plus être.

Ma voix sèche et futile \\
Souffre et surtout \\
Répand l'éloge servile \\
Du moindre de tes atouts.

Chacun a son essence divine \\
Façonnée d'une main d'artiste, \\
Des reflets ravis qui fascinent \\
Et parfois rendent triste.

Fuyant, presque en allé, \\
De ton absence tant étouffé, \\
J'ai le verbe fébrile \\
De cette respiration fragile.

\newpage

Ta présence faisait chanter \\
De ma plume tes milles baisers, \\
Tu avais la beauté fertile \\
Des muses de l'art inutile.
\end{verse}

%\newpage

%%%%%%%%%%%%%%%%%%%%%%%%%%%%%
% LVII. L'AUTEL
%%%%%%%%%%%%%%%%%%%%%%%%%%%%%
\stepcounter{compteur}
\poemtitle{\numero L'Autel}
\settowidth{\versewidth}{J'encenserai le monde entier !}
\begin{verse}[\versewidth]
Demain, à la très-belle \\
Je dédierais un autel, \\
Pour sa gloire et félicité \\
J'encenserai le monde entier !

A jamais, ma très-chère \\
Au rythme de mes vers, \\
Pour que résonne avec clarté \\
Le nom de ma beauté !

%C'est la reine qui, \\
%Le long des gouffres \\
%Amers de l'infortune, \\
%Instaura mon amour \\
%Résigné pour elle \\
%Et régna sur ma vie.
\end{verse}

%\newpage

%%%%%%%%%%%%%%%%%%%%%%%%%%%%%
% LVIII. LA FASCINANTE
%%%%%%%%%%%%%%%%%%%%%%%%%%%%%
%\stepcounter{compteur}
%\poemtitle{\numero La Fascinante}
%\settowidth{\versewidth}{C'est comme si je t'aimais toujours plus tu sais}
%\begin{verse}[\versewidth]
%C'est comme si je t'aimais toujours plus tu sais \\
%Je regarde le ciel chaque jour un peu plus gris \\
%Face à toi qui irradies les astres quand tu ris \\
%C'est comme si ta joie les caressait.

%C'est froid comme les pleurs de la nuit \\
%Dans son écrin soyeux \\
%Que subliment tes beaux yeux \\
%C'est chaud comme l'amour qui suit.

%C'est comme un alcool vaporeux \\
%Un parfum libéré de ta beauté \\
%La volupté d'un hiver à tes cotés \\
%C'est juste l'effet des êtres spiritueux.

%Parfois je me sens flotter en paix \\
%Transporté par une fille fascinante \\
%Je flâne permis les étoiles filantes \\
%C'est juste ta façon de sourire tu sais.
%\end{verse}

\newpage

%%%%%%%%%%%%%%%%%%%%%%%%%%%%%
% LIX. BANG BANG
%%%%%%%%%%%%%%%%%%%%%%%%%%%%%
\stepcounter{compteur}
\poemtitle{\numero Bang Bang}
\settowidth{\versewidth}{Fauchée par le destin, elle est partie.}
\begin{verse}[\versewidth]
Elle s'est éteinte cette nuit, \\
L'étincelle, le soleil, la bougie, \\
La lumière qui éclairait mes récits, \\
Fauchée par le destin, elle est partie.

Sans un bruit, sans un cri, \\
Sans un aveu, sans le savoir, \\
Ô mon amour, mon désespoir, \\
Je t'aimais plus que ma vie !

Mais qui peut prévoir \\
Ce que les cieux préparent \\
Pour élever leurs enfants \\
Aux nuées du néant ?

Si j'avais su le sinistre dessein \\
D'un contre temps assassin \\
Si j'avais pu oser quand même \\
Je te l'aurais dit : je t'Aime !

%C'est bien un cœur aveugle \\
%Que celui qui s'inquiète \\
%Pour une mignonnette \\
%Et ne voit pas son cercueil.
\end{verse}

%\newpage

%%%%%%%%%%%%%%%%%%%%%%%%%%%%%
% LX. LA CHUTE
%%%%%%%%%%%%%%%%%%%%%%%%%%%%%
\stepcounter{compteur}
\poemtitle{\numero La Chute}
\settowidth{\versewidth}{Le ciel devenait plus lourd et les plafonds plus hauts}
\begin{verse}[\versewidth]
Le ciel devenait plus lourd et les plafonds plus hauts \\
A mesure que le jusant de mes rêves et de mes idéaux \\
Se retirait de la grève laissant pour seul vestige \\
Un haut-le-cœur blessant d'un douloureux vertige.
\end{verse}

\newpage

%%%%%%%%%%%%%%%%%%%%%%%%%%%%%
% LXI. MON SEIN S'EMPLIT
%%%%%%%%%%%%%%%%%%%%%%%%%%%%%
\stepcounter{compteur}
\poemtitle{\numero Mon sein s'emplit}
\settowidth{\versewidth}{Qui a le droit de creuser sa vie d'interrogations}
\begin{verse}[\versewidth]
Mon sein s'emplit de milles débris, milles désirs, \\
De milles obstacles impossibles à franchir \\
Pour un cœur si tristement rêveur : \\
Pardonne-moi de ne pas être à ta hauteur !

J'ai l'esprit vide de pensées utiles \\
Et envahit par de sombres regrets, \\
Des fantômes de gestes insatisfaits ; \\
De questions à mon amour fragile :

Qui a le droit de creuser sa vie d'interrogations \\
Sans jamais essayer de la remplir de solutions ?
\end{verse}

%\newpage

%%%%%%%%%%%%%%%%%%%%%%%%%%%%%
% LXII. L'ORBE
%%%%%%%%%%%%%%%%%%%%%%%%%%%%%
\stepcounter{compteur}
\poemtitle{\numero L'Orbe}
\settowidth{\versewidth}{Plongé ravines dans la rêveur}
\begin{verse}[\versewidth]
Une clarté Claire l'opaline \\
A ravit l'écho des collines \\
Plongé ravines dans la rêveur \\
Dans la béatitude mon cœur.
\end{verse}

\newpage

%%%%%%%%%%%%%%%%%%%%%%%%%%%%%
% LXIII. LA PRUDE
%%%%%%%%%%%%%%%%%%%%%%%%%%%%%
\stepcounter{compteur}
\poemtitle{\numero La Prude}
\settowidth{\versewidth}{Ferais douter tous les empires}
\begin{verse}[\versewidth]
A quoi joues-tu ma prude \\
Quand un seul de tes sourires \\
Ferais douter tous les empires \\
Vaciller des certitudes

S'égarer nos généraux \\
Et que tu me laisses dépérir \\
Car un seul de tes sourires \\
Est l'essence du Beau.
\end{verse}