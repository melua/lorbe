\section{\sc{\textit{L'amour un temps de siège}}\hfill}

%%%%%%%%%%%%%%%%%%%%%%%%%%%%%
% XXXIX. CE CœUR QUI AIMAIT
%%%%%%%%%%%%%%%%%%%%%%%%%%%%%
\stepcounter{compteur}
\poemtitle{\numero Ce cœur qui aimait}
\settowidth{\versewidth}{Il n'a pas résisté aux débris}
\begin{verse}[\versewidth]

\emph{\hspace{7em}Il y a tant de morceaux blancs, \\
\hspace{7em}De la vaisselle, de la cervelle \\
\hspace{7em}Et quelques dents de mon enfant. \\}
\attrib{E. Guillevic}

Ce cœur qui aimait \\
Voilà qu'il ralenti \\
Ce cœur qui haïssait la haine \\
C'est la vengeance qu'il crie

Comment peut-il en être ainsi ? \\
Un cœur peut-il renier sa mie ?

Ce cœur qui aimait \\
Il n'a pas fallu une nuit \\
Il n'a pas résisté aux débris \\
Du verre pilé de sa vie.
\end{verse}

\newpage

%%%%%%%%%%%%%%%%%%%%%%%%%%%%%
% XL. OUBLI ET VIE
%%%%%%%%%%%%%%%%%%%%%%%%%%%%%
\stepcounter{compteur}
\poemtitle{\numero Oubli et vie}
\settowidth{\versewidth}{Pourtant c'était là mon rêve}
\begin{verse}[\versewidth]
Deux mois sans te voir \\
Une éternité dans le noir \\
Un battement de paupière \\
Un roulement de rivière

Et je t'ai vue \\
Ma muse au regard surpris \\
Une éloquence impromptue \\
Pour toi aussi

Pourtant c'était là mon rêve \\
Te revoir

Sitôt rentré \\
Besoin d'écrire ta lumière \\
De s'épancher sur le papier \\
Graver mes vers

Donnez moi du courage \\
Pour quitter cette cage \\
Réussir à fermer le livre \\
L'oublier et vivre.
\end{verse}

\newpage

%%%%%%%%%%%%%%%%%%%%%%%%%%%%%
% XLI. J'AI BESOIN D'ELLE
%%%%%%%%%%%%%%%%%%%%%%%%%%%%%
\stepcounter{compteur}
\poemtitle{\numero J'ai besoin d'elle}
\settowidth{\versewidth}{J'ai besoin de ses milles et un atours cachés !}
\begin{verse}[\versewidth]
J'ai besoin d'elle, \\
De son visage comme un soleil, \\
J'ai besoin de son sourire, \\
De ses yeux pour m'éblouir.

J'ai besoin de sa voix, \\
De son regard qui me fait roi, \\
J'ai besoin de ses joues, \\
De ses cheveux qui me rendent fou.

J'ai besoin de ses mains, de ses dents, \\
De sa gorge, de son ventre ondulant, \\
De sa bouche, de son front rayonnant, \\
De son buste, de ses os, de son flanc,

De son nombril, de ses jambes galbées, \\
De son cœur fragile, de ses petits pieds, \\
De ses doigts agiles, de ses ongles nacrés, \\
J'ai besoin de ses milles et un atours cachés !

Je besoin d'elle et je ne sais pas quoi faire, \\
Seulement la regarder me plaire, \\
Fixer sa beauté dans un poème, \\
Sans effleurer l'être que j'aime.
\end{verse}

\newpage

%%%%%%%%%%%%%%%%%%%%%%%%%%%%%
% XLII. NUIT
%%%%%%%%%%%%%%%%%%%%%%%%%%%%%
\stepcounter{compteur}
\poemtitle{\numero Nuit}
\settowidth{\versewidth}{La nuit nous donne un point commun}
\begin{verse}[\versewidth]
Cela paraît anodin \\
Mais du soir au matin \\
Tu exaltes ma fièvre \\
Quand tu rêves

Tu reposes légère yeux clos \\
Tu étouffes mes sanglots \\
Sur l'oreiller le corps félin \\
La nuit nous donne un point commun
\end{verse}

%\newpage

%%%%%%%%%%%%%%%%%%%%%%%%%%%%%
% XLIII. LA BOITE
%%%%%%%%%%%%%%%%%%%%%%%%%%%%%
\stepcounter{compteur}
\poemtitle{\numero La Boite}
\settowidth{\versewidth}{Mon esprit est pareil aux champs battus par les vents}
\begin{verse}[\versewidth]
%Une fleur, un été, un ange est passé \\
%Ses délicats pétales dorés se sont ouverts à l'inconnu \\
%Découvrant son cœur rose au froid de la rue \\
%Une feuille est tombée dans le tourbillon de l'allée

%Emportée au loin elle n'a pas vu la plaie ouverte \\
%D'un bourgeon surpris par une main verte \\
%Qui devait finir d'achever \\
%La belle prose à son chevet

%C'est ``des vents le plus redoutable \\
%S'apprend avec philosophie \\
%Et il n'est de tempête insurmontable \\
%Qui empêche la décousue de la vie''

Caché dans une boîte aux pans d'or, \\
Le reste misérable d'un champ dort. \\
%Attend qu'une main sans éventail \\
%Fonde dans un baiser sa maille

%Qu'elle revêt la chaîne brisée \\
%La rivière des amants \\
%La fierté que ces gens \\
%Avaient pris pour eau de rosée

%Mon cœur est comme une prose aux devants \\
%Mon esprit est pareil aux champs battus par les vents \\
Cette graine, il ne tient qu'à toi de l'arroser, \\
Faire jaillir des louanges à mon amour rossé.

Alors un long Nil dévoilera sa source \\
Aux yeux des pays frontaliers \\
Et se flattera dans sa course \\
De n'avoir que toi pour alliée.
\end{verse}

\newpage

%%%%%%%%%%%%%%%%%%%%%%%%%%%%%
% XLIV. L'AMOUR EN SIEGE
%%%%%%%%%%%%%%%%%%%%%%%%%%%%%
\stepcounter{compteur}
\poemtitle{\numero L'Amour en siège}
\settowidth{\versewidth}{Au charme de la peine, aux prunelles sucrées}
\begin{verse}[\versewidth]
Toi la mendiante, la va-nu-pieds \\
Au regard blessant, aux cheveux légers \\
Toi la lycéenne, la fraîche maturité \\
Au charme de la peine, aux prunelles sucrées

Toi l'étudiante, la pieuse avidité \\
Au visage de l'attente, aux pommettes rosées \\
Toi la vedette, la voix incarnée \\
Au sourire de la fête, au cran de beauté

Toi qui m'as pris au piège \\
Toi que j'ai pris pour cible \\
Comme l'amour un temps de siège \\
Tu m'es inaccessible.
\end{verse}

%\newpage

%%%%%%%%%%%%%%%%%%%%%%%%%%%%%
% XLV. DERNIER POEME EN TON NOM
%%%%%%%%%%%%%%%%%%%%%%%%%%%%%
%\stepcounter{compteur}
%\poemtitle{\numero Dernier poème en ton nom}
%\settowidth{\versewidth}{Je t'aime, je t'aime, oui je t'aime !}
%\begin{verse}[\versewidth]
%Je t'aime, je t'aime, oui je t'aime ! \\
%C'est docile un cœur quand t-il aime

%Marion, Marion, Marion ! \\
%C'est doux comme résonne ce nom

%Aime-moi, Aime-moi, Aime-moi ! \\
%C'est dur comment défilent les mois
%\end{verse}

\newpage

%%%%%%%%%%%%%%%%%%%%%%%%%%%%%
% XLVI. L'ARCHER
%%%%%%%%%%%%%%%%%%%%%%%%%%%%%
\stepcounter{compteur}
\poemtitle{\numero L'Archer}
\settowidth{\versewidth}{L'amour, l'entente ou l'oubliette.}
\begin{verse}[\versewidth]
Je cherche dans leur regard \\
Un clin d'œil sous le fard \\
Et dans le cœur des jeunes reines \\
J'épuise mes espérances vaines!

Face à leurs yeux je suis l'archer \\
Soldat que la peur fait trébucher \\
Et ma flèche se brise à leurs cils \\
Sa victoire ne tenait qu'à un fil!

Peu importe maintenant l'ennemi, \\
Sœur d'arme à l'arbalète, \\
Simple dame ou amie, \\
L'amour, l'entente ou l'oubliette!
\end{verse}

%\newpage

%%%%%%%%%%%%%%%%%%%%%%%%%%%%%
% XLVII. LE CœUR DANS LA TOMBE
%%%%%%%%%%%%%%%%%%%%%%%%%%%%%
\stepcounter{compteur}
\poemtitle{\numero Le cœur dans la tombe}
\settowidth{\versewidth}{Tous ces fastes, ces vœux sincères}
\begin{verse}[\versewidth]
J'ai les fêtes en horreur \\
Tout cet amour, ce bonheur \\
Tous ces fastes, ces vœux sincères \\
Toutes ces passions m'exaspèrent

Les débordements des festivités \\
Ne me provoquent que l'inimité \\
Tout palpite et de joie pleure \\
A ces fêtes qui m'écœurent.
\end{verse}

\newpage

%%%%%%%%%%%%%%%%%%%%%%%%%%%%%
% XLVIII. LA RIVIERE
%%%%%%%%%%%%%%%%%%%%%%%%%%%%%
\stepcounter{compteur}
\poemtitle{\numero La rivière}
\settowidth{\versewidth}{Où la clarté de ma brune décline puis s'évanouit.}
\begin{verse}[\versewidth]
Tu étais là, douceur et patience, \\
J'ai vu tes fastes, ta beauté et ton bonheur, \\
J'ai vu ton regard s'éterniser au fil des heures, \\
Et je t'ai donné un nom : Espérance.

Rien ne t'arrêtera, comme le jour suit la nuit, \\
Comme l'arbre torturé tend vers le ciel, \\
Comme l'astre prend des couleurs vermeilles, \\
Et je chanterai comme tu espères ta vie :

La folle rivière qui dans son lit emporte, \\
Quand elle est au zénith, \\
Le fol amour en flots qui te transportent, \\
Quand elle passe trop vite !


Mais une pâle insomnie de brumes \\
A envahi au clair de lune mon fleuve tarit \\
Du froid sec et minéral de l'amertume \\
Où la clarté de ma brune décline puis s'évanouit.

Les Espérances grandissent tellement vite ! \\
Une rose que la beauté félicite, \\
Si sublime dans son manteau impeccable, \\
Brûlait mes yeux d'un amour improbable.
\end{verse}

\newpage

%%%%%%%%%%%%%%%%%%%%%%%%%%%%%
% XLIX. HYMNE
%%%%%%%%%%%%%%%%%%%%%%%%%%%%%
\stepcounter{compteur}
\poemtitle{\numero Hymne}
\settowidth{\versewidth}{Transcendante vision de tendresse,}
\begin{verse}[\versewidth]
Divinité ange et chasseresse, \\
Eparpilleuse des trêves du cœur, \\
Arche et louange de douceur, \\
Transcendante vision de tendresse,

Tu es mon cri d'allégresse, \\
Tu étincelles de toute ton âme, \\
Tu émerveilles et je me pâme, \\
Tu es l'amour de ma jeunesse,

Tu es partout où je ne veux pas, \\
Tu défies ma conscience, \\
Tu décimes ma confiance, \\
Tue, passe ma vie à trépas.
\end{verse}

\newpage

%%%%%%%%%%%%%%%%%%%%%%%%%%%%%
% L. HIVER SANS COULEURS
%%%%%%%%%%%%%%%%%%%%%%%%%%%%%
\stepcounter{compteur}
\poemtitle{\numero Hiver sans couleurs}
\settowidth{\versewidth}{D'un destin ruiné par les gestes qu'on n'a pas faits ?}
\begin{verse}[\versewidth]
Qui voudrait d'un futur bâtit sur des regrets \\
D'un destin ruiné par les gestes qu'on n'a pas faits ? \\
Qui voudrait d'un cœur ravagé par les flammes \\
D'un sort verdi par mon âme ?

%Qui voudrait d'une plaie violette ? \\
%Qui voudrait de mes bleus au sang ? \\
%Qui voudrait de mes maux rougeoyants ? \\
%Qui voudrait d'un sentiment violenté ?

Qui voudrait d'un homme sans feu \\
Sans amour pour éclairer son cœur ? \\
Qui voudrait d'un malheureux \\
Hiver sans couleurs ?

%Qui voudrait d'un ciel sans soleil ? \\
%Qui voudrait d'un nuage sans l'argent ? \\
%Qui voudrait d'un espoir grisonnant ? \\
%Qui voudrait d'une vie sans l'or ?

Qui voudrait du rêve en noir et blanc \\
D'un fusain oublié et sans vie ? \\
Qui voudrait de l'art incandescent \\
De l'œuvre engendrée par mes cris ?
\end{verse}

\newpage

%%%%%%%%%%%%%%%%%%%%%%%%%%%%%
% LI. QUI QU'ELLE SOIT
%%%%%%%%%%%%%%%%%%%%%%%%%%%%%
\stepcounter{compteur}
\poemtitle{\numero Qui qu'elle soit}
\settowidth{\versewidth}{Qu'elle râle, se fasse entendre, mon introuvable !}
\begin{verse}[\versewidth]
Parfois je croise le bras d'un ange \\
A la peau douce, aux yeux de soie ! \\
Mais je ne prends jamais le fer des louanges \\
Par amour, c'est une errance, un poids !

Quelle est cette monotonie envahissante \\
Qui m'entoure d'une tristesse inconsolable ? \\
Quelle est cette solitude insolente \\
Qui me plonge dans un désert de sable ?

Qui est cette inconnue charmante \\
Qu'elle cache à mes yeux misérables ? \\
Qui est cette victorieuse amante \\
Qu'elle fuira d'un pas coupable ?

Qu'elle chasse la peur écrasante \\
Qui assommait mes nuits semblables, \\
Qu'elle écœurait d'une absence démente \\
Qui m'affolait, imperturbable !

%Qui est cette personne insouciante \\
%Qu'elle protège de tout son râble ? \\
%Qui est cette essentielle impatiente ? \\
%Qu'elle râle, se fasse entendre, mon introuvable !

Qu'elle tempête et se batte la vaillante \\
Qui me délivrera de l'enclos condamnable ! \\
Qu'elle voit la blessure saillante \\
Qui étincelle de pitoyable !

Qu'il vienne le svelte archange, \\
Terrasser l'habitude aux abois ! \\
Avant que la folie ne me mange, \\
Par amour qu'elle foudroie, qui qu'elle soit !
\end{verse}

\newpage

%%%%%%%%%%%%%%%%%%%%%%%%%%%%%
% LII. ECHECS
%%%%%%%%%%%%%%%%%%%%%%%%%%%%%
\stepcounter{compteur}
\poemtitle{\numero Echecs}
\settowidth{\versewidth}{Moi l'acteur sans scène fais de toi ma reine !}
\begin{verse}[\versewidth]
Sans but et sans vie quoi que sans soucis, \\
Avancer d'un pas -- le geste est las, \\
L'absence pérenne -- est-ce bien la peine ?

Sans savoir jouer sur l'échiquier, \\
Ni jamais connaître les règles de l'être, \\
Moi l'acteur sans scène fais de toi ma reine !
\end{verse}

%\newpage

%%%%%%%%%%%%%%%%%%%%%%%%%%%%%
% LIII. LE VERT PARADIS DES AMOURS PERDUS
%%%%%%%%%%%%%%%%%%%%%%%%%%%%%
\stepcounter{compteur}
\poemtitle{\numero Le vert paradis des amours perdus}
\settowidth{\versewidth}{Ingrid est tombée sous le feu de la passion}
\begin{verse}[\versewidth]
Ingrid est tombée sous le feu de la passion \\
Sur l'herbe verte du printemps \\
Son corps langoureux est en perdition \\
La nature caresse son dos nonchalant

Au dessus d'elle les cieux \\
Contemplent les gestes d'affection \\
Qui lient ces deux amoureux \\
Un air de désapprobation.
\end{verse}

%\newpage

%%%%%%%%%%%%%%%%%%%%%%%%%%%%%
% LIV. TOMBE POUR UN SOUVENIR
%%%%%%%%%%%%%%%%%%%%%%%%%%%%%
\stepcounter{compteur}
\poemtitle{\numero Tombé pour un souvenir}
\settowidth{\versewidth}{C'est par le passé que mon présent respire}
\begin{verse}[\versewidth]
Tombé d'amour pour un souvenir \\
C'est par le passé que mon présent respire \\
Et ses saveurs sucrées déjà fanées \\
Se mêle au goût amer de mes journées.
\end{verse}