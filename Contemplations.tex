\section{\sc{\textit{Les contemplations}}\hfill}

%%%%%%%%%%%%%%%%%%%%%%%%%%%%%
% XXI. L'HEUREUX PERDANT
%%%%%%%%%%%%%%%%%%%%%%%%%%%%%
\stepcounter{compteur}
\poemtitle{\numero L'Heureux perdant}
\settowidth{\versewidth}{Car malgré tout, à jouer avec elle, je gagne toujours :}
\begin{verse}[\versewidth]
Elle a cette aisance quand elle se déplace \\
Cette façon de se mouvoir dans l'espace \\
Qui fait de chacun de ses mouvements \\
Un délice pour les sens au firmament.

Qu'elle perde ou qu'elle triomphe le plus souvent \\
N'a pas d'importance pour moi maintenant \\
Car malgré tout, à jouer avec elle, je gagne toujours : \\
C'est si beau de voir son sourire sans détours...
\end{verse}

%\newpage

%%%%%%%%%%%%%%%%%%%%%%%%%%%%%
% XXII. L'AVORTE
%%%%%%%%%%%%%%%%%%%%%%%%%%%%%
\stepcounter{compteur}
\poemtitle{\numero L'Avorté}
\settowidth{\versewidth}{C'est pour être plus près de toi}
\begin{verse}[\versewidth]
%Par deux fois tu m'as donné la vie, j'en ai fais des poèmes. \\
%Et après ? Tous ces meurtres qu'en ton nom le vent sème... \\
%Je cherchais un titre à ce poème que je n'avais nommé \\
%Mais désormais il est tout trouvé : L'Avorté.

%Comme toutes mes entreprises amoureuses, \\
%Celui-là n'aura de fin que dans l'adversité... \\
%L'issue de ma vie aurait pu être heureuse \\
%Si elle s'était arrêtée où elle avait commencé :

Je suis pour toi ce que \\
Hippolyte est à Aricie \\
A la différence que \\
J'ignore la fin du récit


Et si je me dérobe à tes yeux \\
C'est pour être plus près de toi \\
Aussi translucide que je sois \\
Derrière tes éternels cheveux.
\end{verse}

\newpage

%%%%%%%%%%%%%%%%%%%%%%%%%%%%%
% XXIII. LE REFUGIE
%%%%%%%%%%%%%%%%%%%%%%%%%%%%%
\stepcounter{compteur}
\poemtitle{\numero Le Réfugié}
\settowidth{\versewidth}{Dans l'estime de ceux qui m'ont jadis soutenu.}
\begin{verse}[\versewidth]
Je me suis levé ce matin, \\
L'air était froid et le ciel magnifique. \\
Peut-être est-ce mon destin, \\
Contempler -- transit -- l'onirique.

Je me suis couché ce soir \\
L'air était froid et le ciel roux. \\
Peut-être ai-je été fait pour choir \\
Et ne connaître -- de toi – rien de doux.

Je me suis réfugié pour écrire : \\
Par ma faute je me suis perdu \\
Et je crains à jamais de ne pouvoir revenir \\
Dans l'estime de ceux qui m'ont jadis soutenu.
\end{verse}

\newpage

%%%%%%%%%%%%%%%%%%%%%%%%%%%%%
% XXIV. LA DESCENTE
%%%%%%%%%%%%%%%%%%%%%%%%%%%%%
\stepcounter{compteur}
\poemtitle{\numero La Descente}
\settowidth{\versewidth}{Sur les faïences éparpillées}
\begin{verse}[\versewidth]
Ma fontaine brisée \\
Sur les faïences éparpillées \\
Ma jouvence se répand \\
Sur le sol de sang

Mon amour martyrisé \\
Sur le glorieux autel \\
Ma métamorphose terminée \\
Sur la chute des ailes

Ma réalité éclatée \\
Sur le mur des pendaisons \\
Ma fébrile passion \\
Sur le cachet reposée

Mon salut de toujours \\
Face au désespérant jour : \\
Ma poésie renaissante \\
Forge l'âme puissante.
\end{verse}

\newpage

%%%%%%%%%%%%%%%%%%%%%%%%%%%%%
% XXV. LA RAISON PERDUE
%%%%%%%%%%%%%%%%%%%%%%%%%%%%%
\stepcounter{compteur}
\poemtitle{\numero La Raison perdue}
\settowidth{\versewidth}{Et j'espère te voir baigner dans les maux,}
\begin{verse}[\versewidth]
Ces quelques mots résonnent dans ma tête \\
Cette prose que tu m'as accordée, \\
Parmi tant de refus, de requêtes, \\
Il n'a pourtant pas fallu te supplier !

Ces quelques mots dans ma tête résonnent \\
Comme une symphonie tragique, \\
Une folie sublime qui déraisonne, \\
Ö -- mon âme -- l'astre pathétique !

Dans ma tête raisonnent ses quelques mots \\
Cette phrase de liberté qui me sauva, \\
Et j'espère te voir baigner dans les maux, \\
Quand l'amour t'annoncera qu'il s'en va !
\end{verse}

\newpage

%%%%%%%%%%%%%%%%%%%%%%%%%%%%%
% XXVI. LE REGARD
%%%%%%%%%%%%%%%%%%%%%%%%%%%%%
\stepcounter{compteur}
\poemtitle{\numero Le Regard}
\settowidth{\versewidth}{Et j'espère te voir baigner dans les maux,}
\begin{verse}[\versewidth]
Je n'oublierais pas ce regard que tu m'as jeté \\
Après minuit sous la lumière du corridor, \\
Quand les amours filaient l'été \\
Nous pensions à d'autres trésors.

Je n'oublierais jamais ta façon de danser \\
Sous les nuées électroniques, \\
Ta peau délicatement dorée \\
Et tes robes magiques.

Un seul et dernier regard \\
A fait naître en moi tant d'histoires, \\
Plus que la terre ne pourrait contenir \\
Pas assez pour un si beau souvenir.
\end{verse}

\newpage

%%%%%%%%%%%%%%%%%%%%%%%%%%%%%
% XXVII. LA VIE MINERALE
%%%%%%%%%%%%%%%%%%%%%%%%%%%%%
\stepcounter{compteur}
\poemtitle{\numero La vie minérale}
\settowidth{\versewidth}{Mon cœur ne ferait qu'un tour et je délaisserais mon existence}
\begin{verse}[\versewidth]
S'il m'était donné de quitter ma vie pour celle de mon choix, \\
Mon cœur ne ferait qu'un tour et je délaisserais mon existence \\
Pour éprouver la trépidante effusion des sens \\
De ce joyau frôlant son écrin du bout des doigts ;

Pour me sentir ballotté par la chaleureuse aventure \\
De ce fier bijou trônant du haut de sa généreuse chaire. \\
Mon sang ne ferait qu'un tour et je choisirais la douce terre \\
De cette heureuse pierre enchaînée sur tes deux seins purs.
\end{verse}

%\newpage

%%%%%%%%%%%%%%%%%%%%%%%%%%%%%
% XXVIII. A CELLE QUI EST TROP PRETE
%%%%%%%%%%%%%%%%%%%%%%%%%%%%%
\stepcounter{compteur}
\poemtitle{\numero A celle qui est trop prête}
\settowidth{\versewidth}{Mon cœur ne ferait qu'un tour et je délaisserais mon existence}
\begin{verse}[\versewidth]
Prêt ? -- On n'est jamais trop prêt \\
Même au plus près du mur qu'on franchit à deux, \\
Le corps brossant à grands traits \\
Les songes rêvés à deux qui atteignent les cieux.

Aussi -- dans la profondeur de la nuit \\
Les miradors de l'esprit veillent sans répit, \\
Assis au chevet de nos cœurs \\
Et de notre conscience impulsive sa sœur.
\end{verse}

\newpage

%%%%%%%%%%%%%%%%%%%%%%%%%%%%%
% XXIX. L'ENVAHISSEUR
%%%%%%%%%%%%%%%%%%%%%%%%%%%%%
\stepcounter{compteur}
\poemtitle{\numero L'Envahisseur}
\settowidth{\versewidth}{Ah ! Est-ce qu'on peut ne pas douter quand on aime ?}
\begin{verse}[\versewidth]
Ah ! \\
Cette pierre ! \\
Emporte-moi par \\
Monts et merveilles \\
Où je ne douterais plus \\
Pour chaque pas que je fais \\
Pour chaque jeu que je joue \\
Pour chaque parole que je prononce \\
Pour chaque mouvement qui m'anime \\
Quand je t'aime pour chaque seconde passée \\
Ah ! Est-ce qu'on peut ne pas douter quand on aime ?
\end{verse}

%\newpage

%%%%%%%%%%%%%%%%%%%%%%%%%%%%%
% XXX. JE NE VIS QUE POUR SON REGARD
%%%%%%%%%%%%%%%%%%%%%%%%%%%%%
\stepcounter{compteur}
\poemtitle{\numero Je ne vis que pour son regard}
\settowidth{\versewidth}{Je suis l'esclave désireux, l'heureux servant,}
\begin{verse}[\versewidth]
Je ne vis que pour son regard, \\
Ses deux lumières au loin portant \\
Leurs scintillements de phare \\
Éclairant les voies nouvelles du firmament

Je ne songe plus, je vis le moment, \\
Ce bel instant matinal où ton pouvoir \\
Triomphe dans l'asservissement \\
De tous mes fols espoirs

Je suis l'esclave désireux, l'heureux servant, \\
L'enchaîné libéré pour les voir \\
Ces deux pupilles noires aimant \\
Transporter mon cœur jusqu'au soir…
\end{verse}

\newpage

%%%%%%%%%%%%%%%%%%%%%%%%%%%%%
% XXXI. J'AURAIS VOULU
%%%%%%%%%%%%%%%%%%%%%%%%%%%%%
\stepcounter{compteur}
\poemtitle{\numero J'aurais voulu}
\settowidth{\versewidth}{C'est vrai, j'ai plongé souvent dans ses yeux éblouissants}
\begin{verse}[\versewidth]
J'aurais pu, tandis qu'elle était en faible compagnie, \\
Tenter un premier pas, le premier de ma vie, \\
J'aurais été veule et paraîtrait de la folie \\
Mais mon cœur est pur et mon courage trop indécis.

C'est vrai, j'ai plongé souvent dans ses yeux éblouissants \\
Depuis la crique de mon infortune \\
Au travers les récifs amers de ma solitude. \\
Toujours au loin, elle est mon phare, mon soleil levant.

J'aurais pu quitter ma baie, partir la rejoindre, \\
Mais je ne l'ai pas fait, la peur m'a vaincu, \\
Il ne me reste que mes larmes pour me plaindre \\
De ne pouvoir me satisfaire de la simple vue.

Tu parais si jeune derrière tes éclats dorés, \\
Merveilleuse chaîne blonde que je vois en rêve, \\
Tu parais si belle derrière ta frêle beauté \\
Mais mon cœur est un guerrier qui ne connaît de trêve :

Je le sens marteler mon âme sa geôlière, \\
Lever les poings dans le vide, \\
Réprimander la garde livide. \\
Entre l'enchaîné et la peureuse c'est l'éternelle guerre.

%Le prisonnier qui souffre la pâleur des fers \\
%Pleure comme un nouveau né de ne pouvoir rien faire : \\
%La vie est là, qui lui tend les bras, et sa volonté reine \\
%Fléchit sous le regard de celle qu'il aime...
\end{verse}

\newpage

%%%%%%%%%%%%%%%%%%%%%%%%%%%%%
% XXXII. L'APPARITION
%%%%%%%%%%%%%%%%%%%%%%%%%%%%%
\stepcounter{compteur}
\poemtitle{\numero L'apparition}
\settowidth{\versewidth}{Puis lorsque les limbes de la nuit m'eurent dissipé l'esprit}
\begin{verse}[\versewidth]
Au début il y eut un regard \\
Puis lorsque les limbes de la nuit m'eurent dissipé l'esprit \\
Apparu un pâle flambeau éternisant le soir \\
Et par cette face enchanteresse se confondit l'épris ;

De cette vasque abondante surgit un corps onirique \\
Qui des yeux tenta l'héroïque \\
Et d'une silhouette exquise enivra l'obscurité de lumière \\
Avant de disparaître dans un éclair...
\end{verse}

%\newpage

%%%%%%%%%%%%%%%%%%%%%%%%%%%%%
% XXXIII. J'AIMERAIS
%%%%%%%%%%%%%%%%%%%%%%%%%%%%%
\stepcounter{compteur}
\poemtitle{\numero J'aimerais}
\settowidth{\versewidth}{Esquisser tout ton ressentir ;}
\begin{verse}[\versewidth]
J'aimerais savoir te dessiner, \\
Garder ton image prisonnière ; \\
Encore mieux apprécier \\
Ta frêle beauté princière.

J'aimerais pouvoir t'écouter, \\
Bercer ma vie solitaire ; \\
Te voir triompher \\
De ma précédente ta paire.

J'aimerais te faire sourire, \\
Esquisser tout ton ressentir ; \\
Voir l'hiver frémir \\
Aux échos de ton fin rire.
\end{verse}

\newpage

%%%%%%%%%%%%%%%%%%%%%%%%%%%%%
% XXXIV. LAME DE FOND
%%%%%%%%%%%%%%%%%%%%%%%%%%%%%
\stepcounter{compteur}
\poemtitle{\numero Lame de fond}
\settowidth{\versewidth}{Celle qui jadis m'a fait souffrir !}
\begin{verse}[\versewidth]
Lame noire traîtresse, \\
Destructrice et vengeresse, \\
Tu seras mes yeux pour la punir, \\
Celle qui jadis m'a fait souffrir !

A l'ombre de feu la flamme, \\
La seule chose qui me soit donnée \\
C'est de pouvoir lui pardonner \\
De gré d'être une si belle femme !
\end{verse}

%\newpage

%%%%%%%%%%%%%%%%%%%%%%%%%%%%%
% XXXV. SI SEULEMENT
%%%%%%%%%%%%%%%%%%%%%%%%%%%%%
%\stepcounter{compteur}
%\poemtitle{\numero Si seulement}
%\settowidth{\versewidth}{Il me faudrait l'éternité pour contempler tes infinies beautés}
%\begin{verse}[\versewidth]
%Lorsque la nuit revêt son manteau étoilé, \\
%Me laissant son ombre pour te rêver, \\
%Je m'éternise dans ta clarté.

%Lorsque le jour emplit l'humanité, \\
%M'offrant leur soleil pour te regarder, \\
%Je m'éternise dans l'obscurité.

%Quel que soit l'instant de la journée, \\
%Le temps ne pourra jamais me combler, \\
%Je m'éternise dans l'immobilité.

%Il me faudrait l'éternité pour contempler tes infinies beautés \\
%Si seulement je voulais t'approcher...
%\end{verse}

%\newpage

%%%%%%%%%%%%%%%%%%%%%%%%%%%%%
% XXXVI. LA PLUIE
%%%%%%%%%%%%%%%%%%%%%%%%%%%%%
%\stepcounter{compteur}
%\poemtitle{\numero La pluie}
%\settowidth{\versewidth}{Dont il ne m'est pas donné d'espérer ni de croire.}
%\begin{verse}[\versewidth]
%La nuit tombait à renfort de pluie \\
%Et la musique de Bachelet m'égayait l'esprit \\
%Lorsque m'apparu une vision de fantasmagorie, \\
%Une nymphe autre que toi attendait son taxi.

%Les larmes du ciel ruisselaient sur ses habits \\
%Et ses yeux me fixaient avec appui, \\
%Je restai là comme capturé par son regard, \\
%Ne sachant plus penser ni quoi voir.

%Et soudain, sous cette pluie, j'ai compris : \\
%Je vois à travers le mur de ma vie \\
%Une perle épouser ta peau sans fard \\
%Dont il ne m'est pas donné d'espérer ni de croire.
%\end{verse}

%\newpage

%%%%%%%%%%%%%%%%%%%%%%%%%%%%%
% XXXVII. SOIR D'ENCRE
%%%%%%%%%%%%%%%%%%%%%%%%%%%%%
\stepcounter{compteur}
\poemtitle{\numero Soir d'encre}
\settowidth{\versewidth}{Quand fusionnent les rayons de sang de la cathédrale,}
\begin{verse}[\versewidth]
%C'est bien quand la terre couche le soleil contre son sein, \\
%Quand l'air rougit la sensuelle nature \\
%Vêtue de sa robe noir satin \\
%Et que le temps consume la rupture ;

%C'est bien quand le soleil se baigne dans la Loire, \\
%Quand fusionnent les rayons de sang de la cathédrale, \\
%Blessant en ses pointes l'astre au soir, \\
%Que grandissent les piliers de son piédestal.

Mon âme assouvie se refuse au précipice \\
Et noie autant le courage que le vice \\
Dans les lacs de tes yeux insondables \\
Icône des cœurs intouchables

Là où se termine la mer j'irais me jeter \\
Pour que j'eus connu la profondeur de ta beauté \\
Avant de trouver, cachée dans un coquillage, \\
Ma généreuse Vénus, l'éternité sans âge...
\end{verse}

\newpage

%%%%%%%%%%%%%%%%%%%%%%%%%%%%%
% XXXVIII. LE BOUQUET DE LA PRINCESSE
%%%%%%%%%%%%%%%%%%%%%%%%%%%%%
\stepcounter{compteur}
\poemtitle{\numero Le Bouquet de la princesse}
\settowidth{\versewidth}{Princesse des nuits lunaires aux ombres de mystère}
\begin{verse}[\versewidth]
\emph{\hspace{10em}Et je chantais cette romance \\
\hspace{10em}En 1903 sans savoir \\
\hspace{10em}Que mon amour à la semblance \\
\hspace{10em}Du beau Phénix s'il meurt un soir \\
\hspace{10em}Le matin voit sa renaissance. \\}
\attrib{G. Apollinaire}

Princesse des écumes à la salive qui mord \\
Et taillade les berges de mon cœur \\
Faisant naître à chaque vague de remords \\
Un artifice de feu, de sang et de pleurs

Princesse des nuits lunaires aux ombres de mystère \\
Tu es rentrée dans ma vie comme un flambeau \\
Et j'ai entendu la nature fascinée taire \\
Le frisson des cimes, l'envol du corbeau

Princesse des lumières du fin diamant \\
Tu décuples les couleurs de l'artiste \\
Repousse les limites du fond des océans \\
Et joue de nos vies comme un marionnettiste

Princesse des abymes au regard cambré \\
Princesse des temps immémoriaux \\
Tu respires le parfum des Elysées \\
Tu inspires mon cœur de tendres mots

Princesse des nuées \\
Où mon âme a flotté \\
Princesse émeraude \\
Guérit de la peine qui rôde...
\end{verse}