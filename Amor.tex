\section{\sc{\textit{Amor volat undique}}\hfill}

%%%%%%%%%%%%%%%%%%%%%%%%%%%%%
% IX. ALEAS D'AMOUR
%%%%%%%%%%%%%%%%%%%%%%%%%%%%%
\stepcounter{compteur}
\poemtitle{\numero Aléas d'amour}
\settowidth{\versewidth}{Qui cache dans son halo une étoile qui me sera chère.}
\begin{verse}[\versewidth]
Le sang qui coule dans mes veines \\
N'est pas aussi pur que ses cheveux passion \\
Et mon cœur à cette couleur correspond \\
Battant à rompre l'épreuve vaine.

Les parfums qui volent autour de moi \\
Ne valent pas la rose rouge \\
Et tout mon être respire et bouge \\
Au rythme haletant des tristes mois.

Les nuits et les journées passées \\
Ne me font pas oublier son sourire \\
Seule raison suffisante pour vivre \\
Une valse bleue sous le ciel nuancé.

Le printemps viendra et je pourrai \\
Contempler sa beauté vermeille \\
Fruit des plus sublimes merveilles \\
Qui la fait plus belle qu'elle n'y paraît.
\newpage
Son visage enfin au-delà de tout \\
Me réveillera de ce vaste hiver \\
Source de bien des prières \\
%Et malheureux démon de la toux.
Et de malheureuses toux.

Bonheur empli de rêve \\
Je l'aborderai dans la cour \\
Peiné d'être sans Amour \\
Je déborderai de tendresse sans trêve.

Mais qui sait, comme les plus beaux trésors \\
Sera-t-elle déjà conquise \\
Se jouera-t-elle de moi à sa guise \\
Et comme un honnête voleur, me jettera dehors.

Mais qui sait si cette lumière \\
Qu'aveugle je vois dans mon cœur \\
N'est qu'un magnifique lustre sans cœur \\
Qui cache dans son halo une étoile qui me sera chère.
\end{verse}

\newpage

%%%%%%%%%%%%%%%%%%%%%%%%%%%%%
% X. L'AMOUR VOLE PARTOUT
%%%%%%%%%%%%%%%%%%%%%%%%%%%%%
\stepcounter{compteur}
\poemtitle{\numero L'amour vole partout}
\settowidth{\versewidth}{Tes lèvres s'abreuvent de tout leur saoul.}
\begin{verse}[\versewidth]
Pour des heures de travail sans trêves \\
Quelques minutes pour lui dire je t'aime \\
Un instant pour lui murmurer tes rêves \\
Le cœur noyé par tes plus beaux poèmes.

Dans la cour tellement de monde \\
Mais trop de solitaires \\
On vous entraîne dans la ronde \\
Deux, vous êtes seuls sur Terre.

La tête sur ton épaule \\
Son sein sur ta poitrine \\
Ton cœur a le grand rôle, \\
Porter ta passion divine.

Les yeux dans les mêmes eaux, \\
Vos corps parmi la foule \\
Qui monte en bas descend en haut, \\
Tes lèvres s'abreuvent de tout leur saoul.

La cloche retentit dans notre vie \\
Le monde reprend de plus belle \\
Vous restez immobiles comme sans vie \\
Tu lui diras encore -- tu es très belle.

C'est sûr, entre vous deux \\
Et dansant tel un fou \\
De la terre jusqu'aux cieux \\
L'Amour vole partout.
\end{verse}

\newpage

%%%%%%%%%%%%%%%%%%%%%%%%%%%%%
% XI. JOURS DU NEANT
%%%%%%%%%%%%%%%%%%%%%%%%%%%%%
\stepcounter{compteur}
\poemtitle{\numero Jours du néant}
\settowidth{\versewidth}{Certains prennent ça pour de la liberté,}
\begin{verse}[\versewidth]
%Certains prennent ça pour de la liberté, \\
%Pour moi c'est le néant. \\
%Certains en profitent pour s'amuser, \\
%Pour moi c'est un bain de sang.

%Ils en sont fiers, \\
%Moi j'en ai peur. \\
%Ils vont à la mer, \\
%Moi je meurs.

Pendant deux semaines, \\
Durant une éternité. \\
Pendant qu'ils s'aiment, \\
Durant mes nuits étoilées.

Quand ils sont au ski, \\
Quand ils sont en vacances, \\
Dans mon coin je cris, \\
Je me tords de souffrance.

Je préfère ne pas y penser \\
Mais c'est dur de l'oublier, \\
La beauté des scènes \\
Me rappellent la sienne.
\end{verse}

\newpage

%%%%%%%%%%%%%%%%%%%%%%%%%%%%%
% XII. LE REPOS DU COEUR
%%%%%%%%%%%%%%%%%%%%%%%%%%%%%
\stepcounter{compteur}
\poemtitle{\numero Le repos du cœur}
\settowidth{\versewidth}{Tout épuisé et affaibli par les épreuves de la vie,}
\begin{verse}[\versewidth]
Tout épuisé et affaibli par les épreuves de la vie, \\
Corde sensible qui subit les vibrations de l'ennui, \\
Je m'en irai, porté par les desseins de ma mémoire, \\
Jusqu'au pays où les destins sont moins noirs.

Comme une main chaleureuse, \\
Tendue vers le ciel ; \\
Comme une malle fabuleuse, \\
Tendre comme le miel ;

L'écrin de neige légère \\
Des montagnes immaculées \\
Est un autel solitaire \\
Où mon cœur vient se reposer.
\end{verse}

%\newpage

%%%%%%%%%%%%%%%%%%%%%%%%%%%%%
% XIII. LA RAISON SUFFISANTE
%%%%%%%%%%%%%%%%%%%%%%%%%%%%%
\stepcounter{compteur}
\poemtitle{\numero La raison suffisante}
\settowidth{\versewidth}{J'arrivais à sourire de n'importe quoi.}
\begin{verse}[\versewidth]
Pour qui ? Pour quoi ? \\
Voudrais-je vivre cette vie là ? \\
Pour elle pour moi, \\
Je voulais vivre ça.

Comment vivre sans toi ? \\
Arriverai-je à survivre loin de toi ? \\
Par Amour pour toi, \\
J'arrivais à sourire de n'importe quoi.

\vin J'étais heureux et amoureux de toi.
\end{verse}

\newpage

%%%%%%%%%%%%%%%%%%%%%%%%%%%%%
% XIV. VIVRE LES VERS
%%%%%%%%%%%%%%%%%%%%%%%%%%%%%
\stepcounter{compteur}
\poemtitle{\numero Vivre les vers}
\settowidth{\versewidth}{Pour nager avec toi dans la plus puissante des mers...}
\begin{verse}[\versewidth]
Tous ces vers dans ma tête \\
J'ai envie de les écrire \\
J'ai envie de les vivre \\
Dans ma vie comme une fête

Tous ces sentiments dans mon cœur \\
J'aimerai les partager avec toi \\
J'aimerai du plus profond de moi \\
Te dévoiler les secrets de mes pleurs

Toute la passion dans mes vers \\
Je voudrais qu'elle devienne réelle \\
Dans ton cœur, dans tes veines, ma belle \\
Pour nager dans la plus puissante des mers...
\end{verse}

\newpage

%%%%%%%%%%%%%%%%%%%%%%%%%%%%%
% XV. L'ENTRAVE MALEFIQUE
%%%%%%%%%%%%%%%%%%%%%%%%%%%%%
\stepcounter{compteur}
\poemtitle{\numero L'entrave maléfique}
\settowidth{\versewidth}{Suffit juste de cela et d'un ténébreux personnage,}
\begin{verse}[\versewidth]
Suffit qu'un peu mon cœur vibre à votre vue, \\
Que mon sang comble mon visage \\
D'une terrible couleur soutenue \\
Que j'ai peine à dérober à votre passage,

Suffit qu'un peu je t'aime, \\
Que mon âme soit tournée \\
Vers ton aura idéalisée \\
Que je voudrais dans mes poèmes,

Suffit juste de cela et d'un ténébreux personnage, \\
Plus sombre encore que celui qui tue, \\
Un cruel amoureux qui brise mes voyages \\
En ravissant la belle en qui j'avais cru !
\end{verse}

%\newpage

%%%%%%%%%%%%%%%%%%%%%%%%%%%%%
% XVI. L'ABSENTE
%%%%%%%%%%%%%%%%%%%%%%%%%%%%%
%\stepcounter{compteur}
%\poemtitle{\numero L'absente}
%\settowidth{\versewidth}{Pour nager avec toi dans la plus puissante des mers...}
%\begin{verse}[\versewidth]
%Il est des soirs où je ne sais pourquoi j'écris \\
%Il est des soirs où je ne sais pourquoi je vis \\
%Il est des soirs où je ne crois plus cette vieille ironie.

%Cette sornette monstre que j'entends partout \\
%Ce mal impénétrable qui nous rend tous fous \\
%Cette inspiration divine venue de je ne sais où.

%On aura compris où je veux en venir \\
%Ce mot à devenir sourd que j'ai peine à dire \\
%Ce sans quoi tout est bien pire.
%\end{verse}

%\newpage

%%%%%%%%%%%%%%%%%%%%%%%%%%%%%
% XVII. ETOILES BOMBE ET SANG
%%%%%%%%%%%%%%%%%%%%%%%%%%%%%
\stepcounter{compteur}
\poemtitle{\numero Étoiles bombe et sang}
\settowidth{\versewidth}{Une pluie de feux tombe des nuages volants}
\begin{verse}[\versewidth]
Étoiles bombes et sang \\
Les toits les bombes le sang \\
L'ombre noire apeure les enfants \\
Et tue les parents

Quelques sirènes et la vie bascule \\
Une pluie de feux tombe des nuages volants \\
Fait des corps des orphelins et du sang \\
L'orage passé et déjà le crépuscule.
\end{verse}

%\newpage

%%%%%%%%%%%%%%%%%%%%%%%%%%%%%
% XVIII. LA MUSE DEPASSEE
%%%%%%%%%%%%%%%%%%%%%%%%%%%%%
%\stepcounter{compteur}
%\poemtitle{\numero La muse dépassée}
%\settowidth{\versewidth}{Et s'il fallait faire son éloge, je ne saurais quand terminer}
%\begin{verse}[\versewidth]
%Comme l'amour est étrange, \\
%Comme la beauté est affaire de temps, \\
%Comme elle peut changer un ange, \\
%Comme elle change la rose en chiendent.

%Au lieu de regarder la mer à mes pieds, \\
%Je contemplais cette femme que je n'ose nommer, \\
%Oubliant mon sentiment passé, \\
%Je n'avais de cesse de la comparer :

%De tes prunelles émouvantes \\
%Elle a fait des sphères terrifiantes \\
%De ton nez si finement dessiné \\
%Elle a fait une ébauche condamnée

%De ton sourire qui irradie \\
%Elle a fait une froide insomnie \\
%De ton menton chatouilleux \\
%Elle a fait un lointain jeu

%De ton cou parfumé \\
%Elle a fait un vin tourné \\
%De ta chevelure -- amour \\
%Elle a fait des cheveux trop lourds

%De ta silhouette vénitienne \\
%Elle a fait une démarche ancienne \\
%De ta voix enchanteresse \\
%Elle a fait une exécrable ivresse

%De toutes les choses qui font que je t'aime, \\
%Elle a repoussé les limites en une soirée, \\
%Une si soudaine féminité m'a touché, \\
%Depuis j'ai guéri mes poèmes.

%Je t'ai oublié, toi et tous tes colifichets \\
%Et s'il fallait faire son éloge, je ne saurais quand terminer \\
%Il y avait déjà tellement de choses à dire \\
%Lorsque j'entendais son tendre rire.
%\end{verse}

\newpage

%%%%%%%%%%%%%%%%%%%%%%%%%%%%%
% XIX. LES MAIS
%%%%%%%%%%%%%%%%%%%%%%%%%%%%%
\stepcounter{compteur}
\poemtitle{\numero Les mais}
\settowidth{\versewidth}{Les mets c'est pour les gourmets la Cour.}
\begin{verse}[\versewidth]
Mets -- Il n'y a pas de mets qui tienne, \\
Les mets c'est pour les gourmets la Cour.

Aimer -- Il n'y a personne qui t'aime, \\
Aimer c'est pour les cours mais d'Amour.

L'aimer -- Il n'y a pas de mai que t'aimes, \\
L'aimer c'est pour tous les Jours.

Mais -- Il n'y a plus personne qui tienne, \\
Les mais c'est pour les gourmets d'Amour.
\end{verse}

%\newpage

%%%%%%%%%%%%%%%%%%%%%%%%%%%%%
% XX. LUX AETERNA
%%%%%%%%%%%%%%%%%%%%%%%%%%%%%
%\stepcounter{compteur}
%\poemtitle{\numero Lux Aeterna}
%\settowidth{\versewidth}{Matin me frôle la joue de ses doigts grêles,}
%\begin{verse}[\versewidth]
%Luminis. \\
%Une lumière de la fenêtre du \\
%Matin me frôle la joue de ses doigts grêles, \\
%Instant monotone du quotidien. \\
%Nul être qui me réveille de son sourire, \\
%Instant fragile et privilégié \\
%Sous les feux glacés du soleil.

%Noctis. \\
%Orbe éclatante au loin et espérante \\
%Comme un phare sur une mer \\
%Terrible, corne dans le noir \\
%Invitant à la suivre \\
%Sous les traits brumeux de la nuit.

%Luna. \\
%Unique source dans ma vie, \\
%Native de la passion, \\
%Amour scintilleux.

%Le reste n'a pas d'importance. \\
%Une clarté sublime resplendie de la \\
%Créature qui m'anime dans ces \\
%Instants fragiles \\
%Et rêvés.
%\end{verse}